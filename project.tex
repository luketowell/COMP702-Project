\documentclass[a4paper,11pt]{report}
\usepackage{cite}
\usepackage{graphicx}

%%%%  Add some length to the page, as margins always seem too big.  
\addtolength{\topmargin}{-.5in}

\begin{document}

%%%%  Number the initial front matter with roman numerals
\pagenumbering{roman}


%%%%  TITLE PAGE
%%  Here are the elements that make up the title page of the document.  

\thispagestyle{empty}

\title{\LARGE
ASDA Stocking Support System (Xpire)}

\author{
Luke Michael Towell
\\    \\    \\
A DISSERTATION
\\    \\
Submitted to 
\\    \\    \\ 
The University of Liverpool
\\    \\
\\    \\
in partial fulfilment of the requirements
\\
for the degree of 
\\     \\
MASTER OF SCIENCE
\\     \\    \\    \\
}


%%  Fill in a date here if you want, or comment out the line below and
%%  the current date will be automatically inserted for you.  
\date{}


\maketitle


%%%%  ABSTRACT
\chapter*{\center Abstract}

The aim of the project is to create an application which enables ASDA colleagues to identify and inform colleagues of which items within their fresh departments are going out of date and need to be marked down or could potentially be provided to food shelters. The technology that will be utilised in this project will make use of mobile technology, backend web services and Database stored procedures in order to analyse and inform colleagues which items are likely to go out of date on which date and then informs colleagues to go and reduce the products. This application will be used throughout ASDA stores by colleagues on a daily basis and could produce a significant cost saving and waste reduction.
\\
\\
The final products of this project will be a system of applications which will be ready for deployment into an ASDA store in the future. The requirements and periodic demoing of the produced solution will be provided by ASDA technology colleagues.

\newpage



%%%%  STUDENT DECLARATION ON PLAGIARISM
\chapter*{\center Student Declaration} 

I confirm that I have read and understood the University's Academic Integrity Policy.

I confirm that I have acted honestly, ethically and professionally in conduct leading
to assessment for the programme of study.  

I confirm that I have not copied material from another source nor committed plagiarism
nor fabricated data when completing the attached piece of work.  I confirm that I have 
not previously presented the work or part thereof for assessment for another University
of Liverpool module.  I confirm that I have not copied material from another source, nor
colluded with any other student in the preparation and production of this work.  

I confirm that I have not incorporated into this assignment material that has been 
submitted by me or any other person in support of a successful application for a 
degree of this or any other university or degree-awarding body.  

\vspace*{1in}

\noindent SIGNATURE \verb!______________________________________!

\noindent DATE \hspace*{.4in}  \today

\vspace*{1in}

%%  NOTE ABOUT CONFIDENTIAL MATERIAL  
%%     Students who need to keep their dissertation confidential should uncomment
%%     the following sentence on this same page.  This will preclude the dissertation 
%%     from being placed in the University Library.  Students that submit work that 
%%     isn't confidential should leave this line commented out.  

% This dissertation contains material that is confidential and/or commercially
% sensitive. It is included here on the understanding that this will not be revealed to 
% any person not involved in the assessment process.  


\newpage

%%%%  ACKNOWLEDGMENTS
%%%%    A section for Acknowledgments, should you want one.



\newpage


%%%%  TABLE OF CONTENTS  
%%%%      Usually the following command will give you the formatting you want.  

\tableofcontents


%%%%  Turn page numbering back to arabic.  This also resets the numbering
%%%%  to begin again at page 1.  

\pagenumbering{arabic}

%%%%  INTRODUCTION

\chapter{Project Introduction}\label{chap:intro}

\section{Scope}

\subsection{ASDA}
This project is being completed on behalf of ASDA.
ASDA is one of the four largest supermarkets in the UK stocking a variety of goods including fresh, frozen and ambient food goods, non perishable items such as clothes and home products and various other lines of stock. The main focus of this project will be on the management of fresh stock which has a recent expirary date e.g. fresh fruit and vegetables and chilled meats and dairy goods.

As well as being one of the main supermarkets in the UK, ASDA is also part of the US company Wal-Mart which is at the time of writing listed as the largest company in the world. This affords ASDA some benefits with regards to technology via the use of shared technologies across the globe.

Despite this project being completed on behalf of ASDA the applications and software that will be produced at the end of it will
 be suitable for use by any business which deals with products that have an expiration date or use by date.

\subsection{Problem Statement}\label{sec:problem}

The aim of this project is to design an application and stock management system which is able
 to inform store managers and store colleagues on which items of stock are going out of date or
 close to going out of date on the shop floor. This stock will then either be marked 
 down on the shop floor in order to be sold on the day or will be identified as able to be donated to other charitable
causes within the local community as part of the ASDA commitment to engage and assist in the local community.
\\
\\
The key goals of the developed system are to include:

- An application which will direct colleagues to which items need to be marked down.

- The hours spent in store manually marking down products will be reduced.

- Stock which is wasted will be reduced and identified for redistribution.

\subsection{Potential Expansion}
- Timesheeting
\\
Through the use of data mining from the data collected by the application, we should ba able to accurately predict how long it takes staff to organisae and mark down the expiring items within their respective departments. Once this data has been collected then theoretically it could be used in order to calculate how many staff are needed in order to mark down items on a particular day depending on the number of items that need to be reduced as well as the length of time that it is likely to take the staff in order to mark down or redistribute the expiring stock on the shop floor. 
\\
- Stock Prediction and Inventory Management
\\
By analysing the number of items which are expiring in individual stores we should also be able to perform trend analysis on the data which should highlight items of stock which are repeatedly marked down. This trend analysis could then be used to inform store managers and the ASDA inventory management systems of any over ordering which is occuring and allow for more accurate stock levels to reduce potential waste in the future.
\\

\section{Approach}

\subsection{Project Methodology}
The methodology that I have chosen to implement this project in is a scrum agile methodology which involves short iterations of development spread over the 8 weeks I will be developing my solution in. 
This will include the breaking down of the application development into smaller development tasks which can be completed easily. 
\\
\\
Scrum involves the sizing, estimation and development of the development tasks prior to development. Once completed the developed features will be demoed to the project supervisors and stakeholders in order to gain feedback which can potentially be added to the project backlog and acted on in following sprints.
\\

The project will include the management and development of multiple application features. Each of the features will be detailed and documented on a kanban board which will be managed via the Trello application.



Scrum typically consists of 5 different types of meetings: the stand up, the planning, the Retro and the refinement session. All of these sessions are performed once throughout the sprint in order to plan, agree and highlight any issues that the development team might be having throughout the week.

\subsection{Architectural Diagram}

- The Application will be reachable via 3 methods:
\\
- An android application which is available on in store handsets called TC70, TC70X and TC72. These devices are available to all colleagues who work within ASDA stores and are frequently used for other applications within the store eco system.
\\
- An android and iOS application which will be downloadable and usable from users personal devices
A web browser which is accessible from a managers workbench or a colleague laptop. 

\subsection{Continuous Integration \& Continuous Deployment}

\subsection{Application Breakdown}
As shown in the architectural diagram the application will consist of three seperate applications and 2 databases.
 Below are the descriptions and breakdowns of the applications:

\subsubsection{Mobile Application}
The mobile versions of the application will be written using React-Native which is a modern mobile development
 framework created and maintained by Facebook. The reason for choosing React Native is because of the fact that it is a well known and well documented framework with a large and active user community. It is also the recommended mobile development framework for ASDA and Wal-Mart. 

\subsubsection{Web Application}
The web application will be written using React which is very similar to React-native and has also been widely adopted. There are plenty of libraries for React and it is the chosen development framework because I am already familiar with the React ecosystem. 

\subsubsection{Backend Web Services}
written in Java spring boot which ensures that the services will be enterprise worthy. The services will be hosted within a docker image and will be able to be span up on either Azure application servers or Azure Kubernetes clusters. Object orientated services written using the repository design pattern. All services will be thoroughly unit tested. The webservices will be documented using Swagger so that users and future developers are easily able to pick up and alter the code base. 

\subsubsection{Azure SQL Databases}
SQL data which will be encrypted in transit and at rest manually. This database will contain the item data including sell by and use by dates as well as unique item Identifier and item Name. At the moment I am unsure as to if I will store images of the individual items. If I do then the SQL database will also store the references to the relevant item images.

\subsubsection{Azure Cosmos Database}
The use of Azure Cosmos DB for logging data due to the speed that items can be both written and retrieved from the data stores. Azure Cosmos DB is the Azure non relational database offering with encryption both in transit and at rest. Azure also guarantee an uptime of 99.99999\% 

\subsubsection{Authentication \& Authorisation}
SingleSignOn is the authentication method which I have chosen to adopt for all of my applications. This allows the applications to integrate well within the ASDA technology stack and ensures that colleagues will be able to authenticate with their normal credentials. This removes the need for colleagues to need to remember new credentials or to have to have other accounts created in order to use the Xpire application. 

\subsection{UML}
UML diagram of example task for frequently performed tasks
The UML diagram outlined below shows user interaction of the various applications.

\subsection{Interaction Chart}
The Interaction Chart below hightlights how a user interacts with the planned application and the flow of processes in order to complete the functionality of the user. 
Complete check item information and expiration date.

%%%%  BACKGROUND

\chapter{Background}\label{chap:background}

As part of the preperation of this specification I have conducted an investigation into research that others
 have conducted similar to my project. I have also looked for potentially similar applications which offer 
 similar functionality to users. Below is an outline of the applications and a comparison against the functionality
  that they offer against what my app will capable of.


%%%%  DESIGN

\chapter{Design}\label{chap:design}


%%%%   REFERENCES

%%%%  Section for references, using the \bibitem directive to 
%%%%  specify labels used to cite sources in the document.  

\bibliography{bibliography}{}
\bibliographystyle{plain}

%%%%   APPENDICES
\appendix 


\end{document}
